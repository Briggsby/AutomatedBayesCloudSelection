\documentclass{article}
\usepackage[
top    = 2.75cm,
bottom = 2.50cm,
left   = 3.00cm,
right  = 2.50cm]{geometry}
\usepackage{hyperref}
\usepackage{cite}
\usepackage{setspace}
\usepackage{algorithm}
\title{\vspace{-2.0cm} A Generalizable Framework for Automated Cloud Configuration Selection \\ \vspace{0.5cm} \large Supervisors: Adam Barker \& Yuhui Lin}
\date{2019-06-06}
\author{Jack Briggs - 140011358 \\ MSc Data-Intensive Analysis}
\doublespacing
\begin{document}
\maketitle
\newpage
\section*{Abstract}
Outline of the project using at most 250 words
\newpage
\section*{Declaration}
I declare that the material submitted for assessment
is my own work except where credit is explicitly
given to others by citation or acknowledgement. This
work was performed during the current academic year
except where otherwise stated.
The main text of this project report is NN,NNN* words
long, including project specification and plan.
In submitting this project report to the University of St
Andrews, I give permission for it to be made
available for use in accordance with the regulations of the University Library. I also give permission for the title and abstract to be published and for copies of the report to be made and supplied at cost to any bona fide library or research worker, and to be made available on the World Wide Web. I retain the copyright in this work.
\newpage
\tableofcontents
\newpage
\section*{Introduction}
Describe the problem you set out to solve and the extent
of your success in solving it. You should include the aims
and objectives of the project in order of importance and
try to outline key aspects of your project for the reader to look for in the rest of your report.
\section*{Context Summary}
Surveying the context, the background literature and any
recent work with similar aims. The context survey
describes the work already done in this area, either as
described in textbooks, research papers, or in publicly
available software. You may also describe potentially
useful tools and technologies here but do not go into
project-specific decisions.
\section*{Requirements Specification}
Capturing the properties the software solution must have
in the form of requirements specification. You may wish
to specify different types of requirements and given them
priorities if applicable.
\section*{Software Engineering Process}
The development approach taken and justification for its
adoption.
\section*{Ethics}
Any ethical considerations for the project. You should
scan the signed ethical approval document, and include it
as an appendix. 
\section*{Design}
Indicating the structure of the system, with particular
focus on main ideas of the design, unusual design
features, etc.
\section*{Implementation}
How the implementation was done and tested, with
particular focus on important / novel algorithms and/or
data structures, unusual implementation decisions, novel
user interface features, etc.
\section*{Evaluation and Critical Appraisal}
You should evaluate your own work with respect to your
original objectives. You should also critically evaluate
your work with respect to related work done by others.
You should compare and contrast the project to similar
work in the public domain, for example as written about
in published papers, or as distributed in software available to you. 
\section*{Conclusions}
You should summarise your project, emphasising your
key achievements and significant drawbacks to your
work, and discuss future directions your work could be
taken in.
\newpage
\cite{Alipourfard2017} % ONLY HERE TO STOP ERRORS, DELETE
\bibliographystyle{IEEEtran}
\bibliography{Dissertation}
\newpage
\section*{Appendices}
\subsection*{Testing Summary}
\subsection*{User Manual}
\end{document}